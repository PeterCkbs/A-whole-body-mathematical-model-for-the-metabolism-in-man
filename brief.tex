\documentclass{IEEEtran}
\usepackage{cite}
\usepackage{amsmath,amssymb,amsfonts}
\usepackage{algorithmic}
\usepackage{graphicx}
\usepackage{textcomp}
\def\BibTeX{{\rm B\kern-.05em{\sc i\kern-.025em b}\kern-.08em
    T\kern-.1667em\lower.7ex\hbox{E}\kern-.125emX}}
\begin{document}
\title{A whole-body mathematical model for the metabolism in man}

\author{Jacob Bendsen, Peter Emil Carstensen, John Bagterp Jørgensen
\thanks{J. Bendsen, P.E. Carstensen and J.B. J{\o}rgensen are with Department of Applied Mathematics and Computer Science, Technical University of Denmark, Kgs Lyngby, Denmark. Corresponding author: J.B. J{\o}rgensen (e-mail: jbjo@dtu.dk).}
}

\maketitle

\begin{abstract}
We formulate a mathematical model relating cell metabolism to a whole-body flux model using qualitative knowledge of biochemical pathways. The model consists of 7 organs, 16 metabolites and 29 enzymatic reactions. All reaction rates inside cells is described by Michaelis-Menten kinetics with an addition of a hormonal regulator based on the two hormones insulin and glucagon. This results in a substantial amount of unknown parameters. Methods for determining these parameters are based on energy consumption at rest, metabolite fluxes and concentrations found in literature as well as values reported in similar mathematical models. Ingestion of food is added to the model to simulate metabolite concentrations during the fed-fast cycle. These are compared to safe ranges based on selected literature, and a cyclic homeostasis is achieved during regular intake of food. The model can simulate over several days while still maintaining physiological relevance due to the inclusion of storage forms that can be depleted if food is not ingested regularly. This is a unique feature compared to similar mathematical models. The physiological model incorporating complex cellular metabolism and whole-body mass dynamics can assist development of therapies and control algorithms for a number of metabolic diseases including obesity and diabetes.
\end{abstract}

\begin{IEEEkeywords}
Mathematical modeling, metabolism, systems biology, cyber medical systems, multi-scale modeling
\end{IEEEkeywords}

\section{Introduction}
\label{sec:introduction}

TODO: insert af few references and cite them (such that we get the reference system in place)


\section{Mathematical model}
\label{sec:mathmaticalmodel}

TODO: include diagrams and figures illulstrating the model

Describe main pahtways (glycolysis etc. also have illustrations)

\section{Simulation results}
\label{sec:simulationresults}

\section{Conclusion}
\label{sec:conclusion}


\appendix
\subsection{Model equations}

\subsection{Model parameters}

\end{document}
