%Fra bachelor

A mathematical model relating cell metabolism to a whole body flux model was formulated using inspiration from other mathematical models and current qualitative knowledge of biochemical pathways found in literature. The proposed model consists of 7 organs, 16 metabolites and 29 enzymatic reactions. Michaelis-Menten kinetics was used to describe all reaction rates inside cells, with an addition of a hormonal regulator based on the two hormones insulin and glucagon. This resulted in a substantial amount of unknown parameters. Methods for determining these parameters were based on kcal consumption at rest, metabolite fluxes and concentrations found in literature as well as values reported in similar mathematical models. \\

The model was used to simulate different fed-fast stages and ketogenic diet. The metabolite concentrations are compared to safe ranges based on selected literature. Only the circulating metabolites were thought to match what was found in literature, as information regarding cellular metabolites were sparse. The model had the ability to simulate over the course of several days while still maintaining physiological relevance due to the inclusion of storage forms that can be depleted if food is not ingested regularly. A feature that is unique compared to similar mathematical models. \\

Since the model had the inclusion of insulin and glucagon, a simple way to simulate a type 2 diabetic person was proposed, and it showed similar behaviour as reported in literature. \\

Finally, several ideas on how to extend the model was discussed in the light of future research that could provide valuable insight into how the metabolism in man is connected to cellular responses. In conclusion, a basis for a physiological whole-body model that incorporates complex cellular metabolism was created to assist research in this area.




%Til artikel

A mathematical model relating cell metabolism to a whole-body flux model is formulated using inspiration from other mathematical models and current qualitative knowledge of biochemical pathways found in literature. The proposed model consists of 7 organs, 16 metabolites and 29 enzymatic reactions. All reaction rates inside cells is described by Michaelis-Menten kinetics with an addition of a hormonal regulator based on the two hormones insulin and glucagon. This results in a substantial amount of unknown parameters. Methods for determining these parameters are based on kcal consumption at rest, metabolite fluxes and concentrations found in literature as well as values reported in similar mathematical models. \\

Ingestion of food is added to the model to simulate metabolite concentrations during the fed-fast cycle. These are compared to safe ranges based on selected literature, and a cyclic homeostasis is achieved during regular intake of food. The model can simulate over several days while still maintaining physiological relevance due to the inclusion of storage forms that can be depleted if food is not ingested regularly. A unique feature compared to similar mathematical models. \\

In conclusion, a basis for a physiological model incorporating complex cellular metabolism and whole-body mass dynamics is created to assist research in this area. \\

%Eller

Ideas on how to extend the model are discussed in light of future work that can provide valuable insight into how the metabolism in man is connected to cellular responses. In conclusion, a basis for a physiological model incorporating complex cellular metabolism and whole-body mass dynamics is created to assist research in this area.
